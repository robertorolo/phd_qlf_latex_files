\documentclass[aspectratio=169]{beamer}


\mode<presentation>
{
	\usetheme{Default}
	\usecolortheme{whale}
	\usefonttheme{default}
	\setbeamertemplate{navigation symbols}{}
	\setbeamertemplate{caption}[numbered]
}

%enumerate continuing numbering
\setbeamercovered{highly dynamic}

\newcounter{saveenumi}
\newcommand{\seti}{\setcounter{saveenumi}{\value{enumi}}}
\newcommand{\conti}{\setcounter{enumi}{\value{saveenumi}}}
%

\usepackage[portuguese]{babel}
\usepackage[utf8x]{inputenc}
\usepackage[T1]{fontenc}
\usepackage{natbib}
\usepackage{hyperref}
\usepackage{enumerate}
\usepackage{subfig}
%\usepackage{biblatex}

%numbering
\addtobeamertemplate{navigation symbols}{}{%
	\usebeamerfont{footline}%
	\usebeamercolor[fg]{footline}%
	\hspace{1em}%
	\insertframenumber/\inserttotalframenumber
}

\title[Proposta de tese para o exame de qualificação]{Aprimoramentos em modelagem geológica implícita com funções distância assinaladas}
%\subtitle{Proposta de tese para o exame de qualificação}
\author{Me. Roberto Mentzingen Rolo \\ \small{Orientador: Prof. Dr. João Felipe Coimbra Leite Costa, PhD}}
\institute{Universidade Federal do Rio Grande do Sul \\ Escola de Engenharia \\ Programa de Pós-Graduação em Engenharia de Minas, Metalúrgica e de Materiais}
\date{17 de junho de 2019}

\begin{document}
	
%\section{Apresentação}
	
\begin{frame}
	\titlepage
\end{frame}

%sumário
\begin{frame}{Estrutura}
\begin{scriptsize}
	\tableofcontents
\end{scriptsize}
\end{frame}

\section{Introdução}

\begin{frame}{Introdução}

Construir modelos numéricos de longo, médio e curto prazo para avaliação de recursos/reservas e planejamento de mina exige quatro grandes atividades:

\begin{enumerate}
\item Coleta e gerenciamento de dados;
\item Interpretação e modelagem geológica;
\item Atribuição de teores;
\item Avaliação e gerenciamento da incerteza geológica e de teores.
\end{enumerate}

\end{frame}

\subsection{Interpretação e modelagem geológica}

\begin{frame}{Interpretação e modelagem geológica}

\begin{enumerate}
	\item Identificar diferentes domínios;
	\item Definir os limites de cada função aleatória estacionária.
\end{enumerate}

\begin{figure}[H]
	\begin{center}
		\includegraphics[width=\textwidth]{apresentacao/passo_2}
		\caption{Interpretação e modelagem geológica.}
	\end{center}
\end{figure}

\end{frame}

\subsection{Método tradicional}

\begin{frame}{Metodologia tradicional}

A abordagem tradicional para a criação de modelos geológicos tridimensionais é através da triangulação de polilinhas.

\begin{figure}[H]
	\begin{center}
		\includegraphics[width=0.6\textwidth]{capitulo_1/explicitmodeling}
		\caption{Esquema do método tradicional.}
	\end{center}
\end{figure}

\end{frame}

\begin{frame}{Desvantagens do método tradicional}
	\begin{itemize}
		\item Tedioso e demorado;
		\item Exige um profissional especializado e experiente;
		\item Geometria dos corpos precisa ser simplificada;
		\item Subjetivo;
		\item Não replicável;
		\item Inflexível;
		\item Não avalia a incerteza.
	\end{itemize}
\end{frame}

\subsection{Incerteza do modelo geológico}

\begin{frame}{Incerteza do modelo geológico}

Em muitos casos, a incerteza do modelo geológico pode ser uma fonte de incerteza crucial e deve ser avaliada.

\begin{figure}[H]
	\begin{center}
		\includegraphics[width=0.4\textwidth]{capitulo_1/incerteza_limites}
		\caption{Incerteza do modelo geológico.}
	\end{center}
\end{figure}
\end{frame}

\subsection{Métodos matemáticos}

\begin{frame}{Métodos matemáticos}

	\begin{columns}[t]
		\begin{column}{0.5\textwidth}
			\begin{center}
				\textit{Métodos determinísticos}
			\end{center}
			\begin{itemize}
				\item Vizinho mais próximo;
				\item Krigagem dos indicadores.
			\end{itemize}
		\end{column}
		\begin{column}{0.5\textwidth}
			\begin{center}
				\textit{Métodos estocásticos}
			\end{center}
			\begin{itemize}
				\item Simulação sequencial dos indicadores;
				\item Simulação gaussiana/plurigaussiana truncada;
				\item Simulação multi ponto;
				\item Simulação baseada em objetos;
			\end{itemize}
		\end{column}
	\end{columns}
	
\end{frame}

\subsection{Métodos implícitos}

\begin{frame}{Métodos implícitos}
	\begin{figure}[H]
		\begin{center}
			\includegraphics[width=0.8\textwidth]{capitulo_1/implicit_modelig_pt_1}
			\caption{Esquema dos métodos implícitos.}
		\end{center}
	\end{figure}
\end{frame}

\section{Modelagem geológica implícita com funções distância assinaladas}

\subsection{O banco de dados}

\begin{frame}{O banco de dados}

72 furos totalizando 3349 amostras distribuídas entre 3 diferentes categorias.

	\begin{figure}
		\centering
		\subfloat[Proporções.]{\includegraphics[width=.25\textwidth]{capitulo_2/prop_hist_big.png}}\hspace{20mm}
		\subfloat[Vista das amostras.]{\includegraphics[width=.4\textwidth]{capitulo_2/dados.jpeg}}
		\caption{O banco de dados.}
	\end{figure}
		
\end{frame}

\subsection{Codificando as amostras em indicadores}

\begin{frame}{Codificando as amostras em indicadores}
	\begin{equation}
	i_k(u_\alpha)=\begin{cases}
	1,\:\textrm{se}\:z(u_\alpha)\:\textrm{se pertence ao domínio $k$}\\
	0,\:\textrm{se}\:z(u_\alpha)\:\textrm{caso contrário}\end{cases}
	\label{eq_ind}
	\end{equation}
	
	\begin{figure}[H]
		\caption{Amostras codificadas em indicadores para cada uma das três categorias do banco de dados.} \label{ind}
		\centering
		\subfloat[][Categoria 1]{\includegraphics[width=.3\textwidth]{capitulo_2/inddf1.jpeg}\label{<figure1>}}
		\subfloat[][Categoria 2]{\includegraphics[width=.3\textwidth]{capitulo_2/inddf2.jpeg}\label{<figure2>}}
		\subfloat[][Categoria 3]{\includegraphics[width=.3\textwidth]{capitulo_2/inddf3.jpeg}\label{<figure2>}}
	\end{figure}
\end{frame}

\subsection{Calculando a função distância assinalada}

\begin{frame}{Calculando a função distância assinalada}

\begin{equation}
d_k(u_\alpha)=\begin{cases}
-\parallel u_\alpha-u_\beta\parallel,\:\textrm{se $u_\alpha$ pertence ao domínio}\\
+\parallel u_\alpha-u_\beta\parallel,\:\textrm{se $u_\alpha$ não pertence ao domínio}\end{cases}
\label{eq_mult_sg}
\end{equation}

O local $u_\beta$ corresponde à amostra mais próxima codificada com um indicador diferente de $u_\alpha$.
\begin{figure}[H]
	\caption{\label{2d_ex}Ilustração esquemática mostrando o cálculo das distâncias assinaladas.}
	\begin{center}
		\includegraphics[width=0.3\textwidth]{capitulo_2/2d_ex.jpg}
	\end{center}
	%\legend{Modificado de \citeonline{martin2017implicitmodeling}}
\end{figure}
\end{frame}

\begin{frame}{Calculando a função distância assinalada} 
	\begin{figure}[H]
		\caption{Distâncias assinaladas calculadas para cada uma das categorias do banco de dados.} \label{indcalc}
		\centering
		\subfloat[][Categoria 1]{\includegraphics[width=.3\textwidth]{capitulo_2/df1.jpeg}\label{<figure1>}}
		\subfloat[][Categoria 2]{\includegraphics[width=.3\textwidth]{capitulo_2/df2.jpeg}\label{<figure2>}}
		\subfloat[][Categoria 3]{\includegraphics[width=.3\textwidth]{capitulo_2/df3.jpeg}\label{<figure2>}}
	\end{figure}
\end{frame}

\subsection{Variografia das distâncias assinaladas}

\begin{frame}{Variografia das distâncias assinaladas}

Distâncias assinaladas não são estacionárias, o variograma não se estabiliza em um patamar. Além disso, o caráter extremamente contínuo das distâncias torna a identificação analítica das direções principais um processo embaraçoso.

\begin{itemize}
	\item Treinar o variograma usando validação cruzada;
	\item Tentar modelar interativamente os variogramas experimentais;
	\item Calcular e modelar os variogramas para as propriedades de indicadores e transformá-los em um equivalente gaussiano para as distâncias assinaladas;
	\item inferir um modelo de covariância plausível visualmente a partir das amostras ou de mapas delineados a mão.
\end{itemize}
\end{frame}

\begin{frame}{Variogramas das distâncias assinaladas}
\begin{figure}[H] 
	\caption{Variogramas experimentais das distâncias assinaladas e modelos para cada uma das categorias do banco de dados.} \label{sd_var}
	\centering
	\subfloat[][Categoria 1]{\includegraphics[width=.3\textwidth]{capitulo_2/var_DF_1.png}\label{<figure1>}}
	\subfloat[][Categoria 2]{\includegraphics[width=.3\textwidth]{capitulo_2/var_DF_2.png}\label{<figure2>}}
	\subfloat[][Categoria 3]{\includegraphics[width=.3\textwidth]{capitulo_2/var_DF_3.png}\label{<figure2>}}
\end{figure}
\end{frame}

\begin{frame}{Variogramas dos indicadores}
	\begin{figure}[H] 
		\caption{Variogramas experimentais dos indicadores e modelos para cada uma das categorias do banco de dados.} \label{ind_var}
		\centering
		\subfloat[][Categoria 1]{\includegraphics[width=.3\textwidth]{capitulo_2/var_Ind_DF_1.png}\label{<figure1>}}
		\subfloat[][Categoria 2]{\includegraphics[width=.3\textwidth]{capitulo_2/var_Ind_DF_2.png}\label{<figure2>}}
		\subfloat[][Categoria 3]{\includegraphics[width=.3\textwidth]{capitulo_2/var_Ind_DF_3.png}\label{<figure2>}}
	\end{figure}
\end{frame}

\begin{frame}{Alternativa ao cálculo e modelagem dos variogramas}
	\begin{figure}[H]
		\caption{\label{cov_table}Fluxograma da modelagem geológica implícita usando tabelas de covariância.}
		\begin{center}
			\includegraphics[width=0.7\textwidth]{capitulo_2/cov_table.png}
		\end{center}
		%\legend{\citeonline{kloechner_cov_table}}
	\end{figure}
\end{frame}

\subsection{Interpolação das distâncias assinaladas}

\begin{frame}{Resolução do grid}

\begin{figure}[H]
	\caption{\label{grid_res}Efeito da resolução do \textit{grid} na reprodução de estruturas geológicas.}
	\begin{center}
		\includegraphics[width=0.4\textwidth]{capitulo_2/grid_res.png}
	\end{center}
	%\legend{Fonte: \citeonline{martin2017implicitmodeling}}
\end{figure}

\end{frame}

\begin{frame}{Grids criados}
	\begin{table}[H]
		\centering
		\begin{tabular}{lrr}
			& \multicolumn{1}{l}{Grosso} & \multicolumn{1}{l}{Fino} \\ \hline
			nx & 49 & 97 \\
			ny & 49 & 98 \\
			nz & 51 & 102 \\
			sx & 10m & 5m \\
			sy & 10m & 5m \\
			sz & 4m & 2m \\
			num & 122451 & 969612 \\ \hline
		\end{tabular}
		\caption{Parâmetros dos \textit{grids} de definição dos modelos geológicos implícitos.} \label{grid_def}
	\end{table}
\end{frame}

\begin{frame}{Métodos de interpolação}
	\begin{itemize}
		\item \cite{hosseini_deutsch_iqd} utilizaram inverso da distância;
		\item \cite{silvaenhancedgeomodeling} utilizou krigagem ordinária global;
		\item \cite{rolo_dissertacao} utilizou krigagem ordinária;
		\item \cite{silva_dual} aplicaram \textit{dual kriging};
		\item \cite{boisvert_geomodeling} gerou modelos implícitos através de distâncias assinaladas com anisotropia variável local (\textit{Locally varying anisotropy kriging - LVA});
		\item \cite{manchuck_MLS} propuseram a utilização de mínimos quadrados móveis para incorporar interpretação manual e avaliar incerteza.
	\end{itemize}
\end{frame}

\begin{frame}{Métodos de interpolação}
	\begin{figure}[H] 
		\caption{Interpolação das distâncias calculadas por diferentes métodos.} \label{interpo}
		\centering
		\subfloat[][OK com 40 amostras]{\includegraphics[width=.3\textwidth]{capitulo_2/kt3d40.jpeg}\label{<figure1>}}
		\subfloat[][OK com 100 amostras]{\includegraphics[width=.3\textwidth]{capitulo_2/kt3d100.jpeg}\label{<figure2>}}
		\subfloat[][RBF global]{\includegraphics[width=.3\textwidth]{capitulo_2/rbf.jpeg}\label{<figure2>}}
	\end{figure}
\end{frame}

\begin{frame}{Decomposição do domínio}
\begin{figure}[H]
	\caption{\label{pou}Particionamento.}
	\begin{center}
		\includegraphics[width=0.9\textwidth]{capitulo_2/pou.jpg}
	\end{center}
	%\legend{Fonte: \citeonline{martin_boisvert_review_rbf}}
\end{figure}
\end{frame}

\begin{frame}{Benchmark}

Todos os algoritmos utilizados são da biblioteca GSLib e foram executados em um core i7 7700HQ @ 2.8 GHz com 16 Gb de RAM.

\begin{table}[H]
	\centering
	\resizebox{\textwidth}{!}{%
		\begin{tabular}{lllll}
			Método & Tempo grid grosso & Tempo grid fino & Classificação errônea grosso & Classificação errônea fino \\ \hline
			krigagem global isotrópica & 28min &  & 121 &  \\
			krigagem global anisotrópica & 30min 34s &  & 282 &  \\
			krigagem ordinária anisotrópica (40) & 1min 3s & 38min & 137 & 135 \\
			krigagem ordinária anisotrópica (100) &  & 45min 51s &  & 181 \\
			RBF isotrópico & 21.5s &  & 57 &  \\
			RBF anisotrópico &  & 1min 22s &  & 38 \\
			Particionado RBF anisotrópico &  & 1min 2s &  & 39 \\
			Particionado RBF artefatos & 16.5s &  &  & 29 \\
			LVA OK &  &  & 8 &  \\
			LVA RBF &  &  &  & 8 \\
			Krigagem dos indicadores &  & 33min 27s &  & 2 \\ \hline
		\end{tabular}%
	}
	\caption{\textit{Benchmark} dos diferentes métodos de interpolação.} \label{bench}
\end{table}
\end{frame}

\subsection{Visualização do modelo geológico}

\begin{frame}{Visualização do modelo geológico}
\begin{figure}[H]
	\caption{Iso superfícies para a categoria 1 extraída dos diferentes modelos implícitos.} \label{isosup}
	\centering
	\subfloat[][OK com 40 amostras]{\includegraphics[width=.3\textwidth]{capitulo_2/isokt3d40.jpeg}\label{<figure1>}}
	\subfloat[][OK com 100 amostras]{\includegraphics[width=.3\textwidth]{capitulo_2/isokt3dn100.jpeg}\label{<figure2>}}
	\subfloat[][RBF global]{\includegraphics[width=.3\textwidth]{capitulo_2/isorbf.jpeg}\label{<figure2>}}
\end{figure}
\end{frame}

\begin{frame}{Visualização do modelo geológico}
	\begin{figure}[H]
		\caption{Iso superfície extraída do modelo implícito interpolado por RBF para a categoria 1.} \label{iso_cat1}
		\centering
		\subfloat[][Vista 1]{\includegraphics[width=.5\textwidth]{capitulo_2/iso_cat1_rbf.jpeg}\label{<figure1>}}
		\subfloat[][Vista 2]{\includegraphics[width=.5\textwidth]{capitulo_2/iso_cat1_rbf2.jpeg}\label{<figure2>}}
	\end{figure}
\end{frame}

\begin{frame}{Visualização do modelo geológico}
\begin{figure}[H] 
	\caption{Iso superfície extraída do modelo implícito interpolado por RBF para a categoria 2.} \label{iso_cat2}
	\centering
	\subfloat[][Vista 1]{\includegraphics[width=.5\textwidth]{capitulo_2/iso_cat2_rbf.jpeg}\label{<figure1>}}
	\subfloat[][Vista 2]{\includegraphics[width=.5\textwidth]{capitulo_2/iso_cat2_rbr2.jpeg}\label{<figure2>}}
\end{figure}
\end{frame}

\subsection{Adaptação para múltiplas categorias simultaneamente}

\begin{frame}{Adaptação para múltiplas categorias simultaneamente}

\begin{equation}
i_k(u_\alpha)=\begin{cases}
1,\:\textrm{se}\:z(u_\alpha)=k\\
0,\:\textrm{se}\:z(u_\alpha)\:\textrm{caso contrário}\end{cases} k=1,...,K
\label{eq_mult_ind}
\end{equation}

\begin{equation}
d_k(u_\alpha)=\begin{cases}
-\parallel u_\alpha-u_\beta\parallel,\:\textrm{se}\:i_k(u_\alpha)=1\\
+\parallel u_\alpha-u_\beta\parallel,\:\textrm{se}\:i_k(u_\alpha)=0\end{cases} k=1,...,K
\label{eq_mult_sg}
\end{equation}

\begin{equation}
d_k^*(u)=\sum\limits_{\alpha=1}^n \lambda_\alpha(u)d_k(u_\alpha)\quad k=1,...,K
\label{eq_mult_ok}
\end{equation}

\begin{equation}
i^*(u)=k'\;\text{de modo que}\;d_{k'}^*=min\{d_k^*(u)\}_{k=1}^K
\label{eq_mult_rt}
\end{equation}

\end{frame}

\begin{frame}{Adaptação para múltiplas categorias simultaneamente}
\begin{figure}[H]
	\caption{\label{mult_cat}Esquema para criação de um modelo implícito multi categórico.}
	\begin{center}
		\includegraphics[width=\textwidth]{capitulo_2/mult_cat_legenda.jpg}
	\end{center}
	%\legend{Fonte: Modificado de \citeonline{silvageostatlessons}}
\end{figure}
\end{frame}

\begin{frame}{Adaptação para múltiplas categorias simultaneamente}
\begin{figure}[H]
	\caption{Modelo geológico multi categórico.} 
	\label{multi_cat_rbf}
	\centering
\subfloat[][Seções em XZ do modelo implícito gerado por RBF no \textit{grid} fino.]{\includegraphics[width=.5\textwidth]{capitulo_2/anisofinexz.png}\label{<figure1>}}
\subfloat[][Seções em YZ do modelo implícito gerado por RBF no \textit{grid} fino.]{\includegraphics[width=.5\textwidth]{capitulo_2/anisofineryz.png}\label{<figure2>}}
\end{figure}

\end{frame}

\section{Referências bibliográficas}

\begin{frame}{Referências bibliográficas}
	\bibliographystyle{apa}
	\bibliography{bibliografia}
\end{frame}

\end{document}